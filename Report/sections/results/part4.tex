\subsection {Part 4: Setting up a subversion repository}
Completing part 4, the web server had a fully operational SVN repository with authentication mechanisms
\newline

\noindent
{\bf Q6. Describe the security measures you have undertaken to secure your
repository, and how did that affect the security of your Web Application
(Better? Worse?).}
\newline

\noindent
Using the apache servers built-in authentication methods, the repository is only accessible to users that can provide valid usernames and passwords. The repository is also only accessible over an https connection, which protects the integrity of the communication. This authentication method is vulnerable for bruteforcing but a sufficient password strength makes this kind of attack inefficient. Since we use a SSL encrypted connection to submit this data eavesdropping is also hard to accomplish.
\newline

\noindent
Certainly a solution with authentication via SSL certificates would have offered a stronger security but our group decided to use the mentioned method with good passwords. The reason for this is that one of the group members is travelling a lot and often works on other PCs than his own. Therefore he would have to carry his certificate to any of this devices which probably could be malicious since they are operated by other people. Furthermore in case of using a flash-device containing the certificate there is the danger of losing it.
So basically there would be in our case a higher risk to disclose the certificates unintentionally.
However a weak protection of the subversion system is a huge threat to the webapplication itself. If an attacker would be able to get the access he can read and alter the sourcecode of the application. Reading it allows the attacker to find vulnerabilities much easier and altering could be used for a lot of harmful attacks and things like backdoors. Nevertheless the changes being commited by an attacker have to be updated on the server to affect the website and hopefully the updating person would check the file.
%%Hmm....there is still problems in this section. 


