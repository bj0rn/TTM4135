\subsection {PHP and MySQL}
Our PHP application connected to the MySQL database builds the surface for users visiting our side. Since we developed it by ourself and the webapplication is often the first attack target for people with malicious intend we had to care especially here about the used security mechanisms.

\noindent
{\bf Q5. What kind of malicious attacks is your web application (PHP) vulnerable to? Describe them briefly, and point out what countermeasures you have
developed in your code to prevent such attacks.}
\newline

\noindent
Since the application uses a MySQL database to save certain user data, the application is naturally vulnerable against sql injection attacks, which are used to read, manipulate or delete entries from the database. To prevent those attacks we are using prepared SQL-statements. If an attacker somehow is able to get our database all user passwords are hashed with sha-512 to make it difficult to use this data. Furthermore we are sanitizing all user input via $preg_match()$ with a self-defined whitelist. The allowed signs are digits, all normal characters and the special characters $@,!,.,-,_,.and +$. This whitelist offers also protection against cross site scripting attacks since we are not allowing signs which are mandatory to start such an attack. Cross Site Scripting means that malicious javascript code is injected to the webapplication and gets executed on the clients pc. \cite {sqlinjection, xss}
\newline

\noindent
With protection against cross site scripting we have also improved protection against session hijacking since this kind attacks often rely on such a vulnerability. During a session hijacking attack the information which are used to remember a certain user are stealed and used by an attacker to access the application with the rights of the user. Unfortunately we cannot offer protection against other client side attacks which result in a leak of the session information. Nevertheless we are saving the user ip within the session and validate this to prevent hijacking via the internet, however attacks within the same network are still feasible.
\newline

\noindent
Other common kinds of vulnerabilities in php application are file inclusions and file disclosure vulnerabilities allowing an attacker to add or read files on the server. Our application is secure against them since we are using hardcoded html links instead of the php include function and also a hardcoded filename for the download function.










